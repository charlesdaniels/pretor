\documentclass{article}

\usepackage{pretor}

\title {Pretor User's Manual\\\small{For Students}}
\author {}
\date {}

\cfoot{\texttt{pretor 0.0.1}}

\begin{document}

\maketitle

\tableofcontents

\section{Introduction}

\subsection{What is Pretor?}

Pretor is a grading assistant program which makes programming assignments
easier for students and for instructors. From the perspective of student users
of Pretor, a primary advantage is that Pretor will handle packing up your
assignments for submission. Gone are the days of guessing what format your
professor wants your assignment in, or debating between \texttt{tar cvfz
project1.tgz project1/} and \texttt{tar cvfz project1.tgz project1/*}. When you
pack your submissions with Pretor, you can rest assured that your professor
will have no difficulty grading your code.

\subsection{What is a PSF?}

\textbf{PSF} stands for "Pretor Submission File". These are produced and
interacted with by executing the \texttt{pretor-psf} command (more on that
later). A PSF contains all the code or other materials you may have written for
a given assignment, and also contains metadata such as your user or group ID,
section number, course, semester, and so on. A PSF is readable as a zip file,
so simply keeping the PSF you turned in for a given assignment is a great way
to archive your work for future reference.

\pretoremph{\textbf{Remember} while a PSF file may be read as a zip file, you
should not attempt to create your own using a zip archiving tool - submissions
not generated with the \texttt{pretor-psf} command will be missing information
required for grading.}

\section{Using \texttt{pretor-psf}}

\subsection{Preparing Your Project}

Before you can create a PSF file, you must place a \texttt{pretor.toml} file in
the top-level directory of your project. The \texttt{pretor.toml} file contains
information about how your PSF file should be packed. Your professor will
provide an appropriate \texttt{pretor.toml} for each assignment.

\pretoremph{\textbf{Remember} each assignment will have a different
\texttt{pretor.toml}, be careful not to mix-and match between assignments. In
other words, the \texttt{pretor.toml} for \textit{Assignment 1} can not be
interchanged with the one for \textit{Assignment 2}.}

As an example, consider the project structure shown in figure \ref{fig:before}.
We would place the \texttt{pretor.toml} file in the highest directory of the
project, as a peer to the Makefile and README, as shown in figure
\ref{fig:after}. Keep in mind that this is simply an example for a hypothetical
C-based project; your project structure may be different depending on your
course, professor, language, and development environment.

\begin{multicols}{2}

	\begin{figure}[H]

		\dirtree {%
			.1 ./.
			.2 Makefile.
			.2 README.txt.
			.2 doc/.
			.3 manual.pdf.
			.2 src/.
			.3 main.c.
			.3 utils.c.
			.3 utils.h.
			}

		\caption{Sample project structure before adding
		\texttt{pretor.toml}. \label{fig:before}}

	\end{figure}

	\begin{figure}[H]

		\dirtree {%
			.1 ./.
			.2 Makefile.
			.2 README.txt.
			.2 pretor.toml.
			.2 doc/.
			.3 manual.pdf.
			.2 src/.
			.3 main.c.
			.3 utils.c.
			.3 utils.h.
			}

		\caption{Sample project structure after adding
		\texttt{pretor.toml}. \label{fig:after}}

	\end{figure}

\end{multicols}

\subsection{Generating a PSF}

\pretoremph{\textbf{Hint}: when you see \texttt{monospaced text} prefixed with
a \texttt{\$} symbol, this shows a command that you should run in a terminal.
for example, \texttt{\$ echo hello} means you should type "echo hello" into
your terminal and press enter.  }

Now that you have a project ready to submit with a \texttt{pretor.toml} placed
correctly in the project's root directory, let's generate a submission file.


First, \texttt{cd} into your project directory:

\texttt{\$ cd /path/to/project}

Now, call \texttt{pretor-psf} to pack up your project.

\texttt{\$ pretor-psf --create}

An example output of this command for a user named \texttt{cad} might be:

\begin{verbatim}
INFO: creating PSF...
INFO: reading data from .
INFO: generating metadata...
INFO: writing output...
INFO: PSF written to '/home/cad/Desktop/Fall 2019-CSCE313-1-cad-Assignment 1.psf'
\end{verbatim}

\subsection{Specifying Your Group}

By default, the group ID stored in your assignment file is simply the username
you are logged into your computer with.  In the case of group projects, or if
you need to specify a student ID which is different from the username of the
logged in user, you may do so by specifying the \texttt{--group} flag. For
example, if you were part of the group "group1", you might run the command:

\texttt{\$ pretor-psf --create --group group1}

\pretoremph{\textbf{Hint}: you can heck the group ID that generated a PSF
file with the command \texttt{\$ pretor-psf --metadata --input SOME\_FILE.psf}}

\subsection{Troubleshooting}

\subsubsection{\texttt{ERROR: '$\hdots$/pretor.toml' does not exist, refusing to generate PSF}}

If you encounter this error message, then there are two possibilities:

\begin{itemize}

	\item You have run \texttt{pretor-psf} from the wrong directory,
		\texttt{cd} to your project directory and try again.

	\item You forgot to add an appropriate \texttt{pretor.toml} to your
		project directory.

\end{itemize}

\subsubsection{\texttt{ERROR: $\hdots$ was not specified.}}

This error message suggests that your \texttt{pretor.toml} file is present, but
may be missing information. Make sure that you are using the
\texttt{pretor.toml} provided by your professor, and that you have followed any
special instructions your professor may have provided.

\subsubsection{\texttt{ERROR: output file '$\hdots$' exists, refusing to overwrite}}

This error message indicates that the target output file that
\texttt{pretor-psf} would generate already exists. If you definitely want to
replace this file, you can either move or delete it, or you can use the
\texttt{--force} flag to \texttt{pretor-psf}, which will cause it to overwrite
the file.

\end{document}
