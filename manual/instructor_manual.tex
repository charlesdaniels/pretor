\documentclass{book}

\usepackage{pretor}

\title {Pretor User's Manual\\\small{For Instructors}}
\author {}
\date {}

\cfoot{\texttt{\pretorvers}}

\begin{document}

\maketitle

\tableofcontents

\chapter{Introduction}

\section{What is Pretor?}

Pretor is an automated "grading assistant". It is a program which can help you
manage your student's submissions, your grades and feedback, and enable you to
easily create automation. There are several ways to use Pretor...

\begin{enumerate}[i.]

	\item As a tool for facilitating manual grading. In it's default state,
		Pretor will manage student submissions, allow you to interact
		them in a Bash shell, record your scores in a simple TOML
		format, archive your grades and feedback for posterity, and
		export a spreadsheet you can upload into your university's LMS.

	\item As a platform for machine-assisted grading. It's easy to write
		your own plugins or other tools; you can then use Pretor as a
		tool to interactively orchestrate your automation.

	\item As a library for fully-automated grading. Pretor provides
		powerful primitives that could be used as the basis for an
		unattended grading system. The interactive grading REPL also
		supports the execution of script files, allowing it to be run
		in a headless unattended mode.

\end{enumerate}

There are three major components of Pretor:

\paragraph{\texttt{pretor-psf}} is used by students to generate PSFs (Pretor
Submission Files), which they can submit through whatever mechanism you find
appropriate.

\paragraph{\texttt{pretor-grade}} implements an interactive REPL that enables a
grader to efficiently iterate through many PSFs in sequence.
\texttt{pretor-grade} ultimately produces more PSFs as output, which have the
grades and other feedback the grader assigns "burned in" to them.

\paragraph{\texttt{pretor-export}} is a tool which operates on the PSFs
produced by \texttt{pretor-grade} and generates output files that can be read
by humans, or imported by an LMS for bulk grading.

\section{Understanding the Pretor Data Model}

\begin{figure}[H]

	\centering

	\resizebox{0.8\textwidth}{!}{\digraph{abc}{
		rankdir=LR
		Grade -> Assignment;
		Assignment -> Course;
		Course -> Assignment;
		PSF->Revision;
		Revision->Grade;
		PSF -> "Parent Revision";
		"Parent Revision" -> Revision;
	}}

	\caption{High-level overview of the Pretor data model
	\label{fig:datamodel}}

\end{figure}

Understanding Pretor's data model is critical to make efficient use of its
features. Fortunately, Pretor has a relatively simple data model that aligns
closely with how courses, grades, and submissions are intuitively reasoned
about.

\begin{itemize}

	\item A PSF contains one or more revisions.

	\item A revision may contain zero or one grades.

	\item A grade is associated with an assignment.

	\item An assignment is associated with a course.

	\item A course is associated with one or more assignments.

\end{itemize}

\pretoremph{\textbf{Aside}: for technical reasons, all PSF files contain
serialized copies of the assignment and course information for each grade they
contain. This is because a grade is meaningless without a rubric (to weight
each category score), and a course (to determine the weight of the assignment
overall).}

You as the instructor interact with the data model in several ways. One
important way is by writing a \textbf{course definition file}, which defines
the set of assignments in your course and their relative weights, as well as
the rubric categories for each assignment. This is used by
\texttt{pretor-grade} to compute the scores for each assignment you grade, and
by \texttt{pretor-export} to generate appropriate score values.

You will also interact with the data model by grading assignments. Each time
you grade an assignment, you are creating a \textbf{revision} in the PSF the
student turned in, creating a \textbf{grade}, and attaching the grade to the
revision.

\pretoremph{\textbf{Key Concept}: Pretor has it's own internal revision control
system. Student-generated PSFs contain an initial "submission" revision. When
you grade a PSF, you create a "grade" revision, which can include both changes
to score or other metadata, as well as changes to the student's submitted code.
This is valuable because it makes it easy to track changes made to get student
code to compile, and allows you to make in-line comments within the student's
code. You can even revise an existing grade revision later if you realize you
made a mistake, which would create a third revision. Arbitrarily many grade
revisions may be made.}

\section{Pretor Workflow}

Using Pretor is straightforward, barring additions made by third-party plugins,
a typical Pretor grading workflow looks like this:

\begin{enumerate}

	\item Download PSFs for a specific assignment from your institution's
		LMS

	\item Run \texttt{pretor-grade} on the downloaded files, assigning a
		grade to each, this produces more PSFs which contain both the
		student's original submission and your modifications and
		feedback

	\item Run \texttt{pretor-export} on the PSFs generated in the previous
		steps to generate a CSV file appropriate for upload into your
		LMS

\end{enumerate}

\chapter{Grading With Pretor}

\section{Grading Basics}

You can begin an interactive grading session with the command
\texttt{pretor-grade}. \texttt{pretor-grade} has many useful parameters you
should explore\footnote{see \texttt{pretor-grade --help}}, but the most
important are \texttt{--ingest}, \texttt{--outputdir}, and
\texttt{--coursepath}.

\paragraph{\texttt{--ingest}} is used to specify a directory where you have
downloaded your PSFs. This directory is searched recursively for
\texttt{*.psf}, all of which are loaded into your grading REPL before it
begins. You can also ingest PSFs after launching via the \texttt{ingest}
command.

\paragraph{\texttt{--outputdir}} when you finish grading an assignment and mark
it as finalized, the resulting PSF will be stored in this directory. If
unspecified, they'll be placed in your working directory.

\paragraph{\texttt{--coursedir}} specifies the directory where your course
definition file(s) are stored. When you begin grading a PSF, the course name
and assignment name specified in the submission's \texttt{pretor.toml} are
looked up by recursive search through every TOML in your configured coursedir.
When a matching file is found, it is loaded and used to pre-populate the
\texttt{grade.toml} that you will use to enter your scores. If you don't
specify this, your working directory will be used.

You should be greeted by a prompt that looks like this:

\begin{verbatim}
PRETOR version 0.0.1 interactive grading shell.
grader>
\end{verbatim}

You can enter the \texttt{help} command here to see a list of all commands
available in the REPL, and \texttt{help <command name>} to see documentation
for a specific command.

While there are many useful commands available, the most essential are:

\paragraph{\texttt{loaded}} displays a list of PSF files that have been loaded

\paragraph{\texttt{current}} show information about the PSF you are working on
right now

\paragraph{\texttt{next}} load the next un-graded PSF that is loaded

\paragraph{\texttt{interact}} drop to a Bash shell to grade the PSF

\paragraph{\texttt{showgrade}} show the score card for the current PSF

\paragraph{\texttt{finalize}} save your changes to the PSF and write it out
into the configured output directory

With only these commands, you can perform all grading tasks with Pretor.

Let's look at an example grading session with Pretor:

\begin{verbatim}
$ pretor-grade --ingest submissions
INFO: Loading PSF file 'submissions/Spring 1973-ABC123-2-jsmith-Assignment 1.psf'
INFO: Loading PSF file 'submissions/Spring 1973-ABC123-2-jdeer-Assignment 1.psf'
INFO: Loading PSF file 'submissions/Spring 1973-ABC123-2-jdoe-Assignment 1.psf'
PRETOR version 0.0.1 interactive grading shell.
grader> loaded
       0: Spring 1973-ABC123-2-jsmith-Assignment 1.psf
       1: Spring 1973-ABC123-2-jdeer-Assignment 1.psf
       2: Spring 1973-ABC123-2-jdoe-Assignment 1.psf
       grader> next
<PSF ID=UUID('7c6f7597-aba0-43bd-bee6-a829943bfcd7')>
semester      Spring 1973
section       2
assignment    Assignment 1
group         jsmith
course        ABC123
timestamp     2019-02-06 19:30:33.046311
archive_name  submissions/Spring 1973-ABC123-2-jsmith-Assignment 1.psf
PSF has NOT been graded
grader> interact
INFO: dropping you to a shell: bash --norc
grading Assignment 1 by jsmith $ tree
.
├── grade.toml
└── submission
    ├── doc
    │   └── HOWTO.txt
    ├── hello.c
    ├── Makefile
    ├── pretor.toml
    ├── util.c
    └── util.h

2 directories, 7 files
grading Assignment 1 by jsmith $ cat grade.toml
feedback = ""
bonus_multiplier = 0.0
bonus_marks = 0
bonus_score = 0.0
penalty_multiplier = 0.0
penalty_marks = 0
penalty_score = 0.0
assignment_name = "Assignment 1"

[categories]
correctness = 70
style = 30

\end{verbatim}

The \texttt{grade.toml} file is perhaps the most important thing to notice
here. This is how you input the grade you would like to assign. When you
interact with a PSF for the first time, this file is populated with the maximum
values for each category as determined by your course definition file. In other
words, every submission starts out with a 100\% score, and modifying the values
in the \texttt{[categories]} section allows you to change the submission's
score.

\begin{verbatim}
grading Assignment 1 by jsmith $ exit
exit
INFO: shell session terminated
grader> showgrade
SCORECARD FOR ABC123: Assignment 1

CATEGORY     MARKS  MAX MARKS  PERCENT SCORE
correctness  70     70         100.00%
style        30     30         100.00%

OVERALL MARKS: 100
MAXIMUM OVERALL MARKS: 100
RAW SCORE: 100.00%

OVERALL SCORE: 100.00%

grader> finalize
INFO: writing to 'Spring 1973-ABC123-2-jsmith-Assignment 1.psf'
grader> exit
$ pretor-psf --scorecard --input Spring\ 1973-ABC123-2-jsmith-Assignment\ 1.psf
SCORECARD FOR ABC123: Assignment 1

CATEGORY     MARKS  MAX MARKS  PERCENT SCORE
correctness  70     70         100.00%
style        30     30         100.00%

OVERALL MARKS: 100
MAXIMUM OVERALL MARKS: 100
RAW SCORE: 100.00%

OVERALL SCORE: 100.00%

\end{verbatim}

Notice that the assigned grade is saved out to disk as soon as
\texttt{finalize} is issued, and can be retrieved later using
\texttt{pretor-psf}.

Now consider an example where we have graded several PSFs already, and want to
see the overall score on each. For this, we can use the \texttt{pretor-export}
command:

\begin{verbatim}
$ ls
 coursedefs        'Spring 1973-ABC123-2-jdeer-Assignment 1.psf'    submissions
 README.md         'Spring 1973-ABC123-2-jdoe-Assignment 1.psf'
 sample_solutions  'Spring 1973-ABC123-2-jsmith-Assignment 1.psf'
 $ pretor-export --input '*.psf' --table
Spring 1973  ABC123  2  jsmith  90
Spring 1973  ABC123  2  jdeer   50
Spring 1973  ABC123  2  jdoe    90  submitted late, -10%
\end{verbatim}


\end{document}
