\documentclass{book}

\usepackage{pretor}

\title {Pretor User's Manual\\\small{For Instructors}}
\author {}
\date {}

\cfoot{\texttt{\pretorvers}}

\begin{document}

\maketitle

\tableofcontents

\chapter{Introduction}

\section{What is Pretor?}

Pretor is an automated "grading assistant". It is a program which can help you
manage your student's submissions, your grades and feedback, and enable you to
easily create automation. There are several ways to use Pretor...

\begin{enumerate}[i.]

	\item As a tool for facilitating manual grading. In it's default state,
		Pretor will manage student submissions, allow you to interact
		them in a Bash shell, record your scores in a simple TOML
		format, archive your grades and feedback for posterity, and
		export a spreadsheet you can upload into your university's LMS.

	\item As a platform for machine-assisted grading. It's easy to write
		your own plugins or other tools; you can then use Pretor as a
		tool to interactively orchestrate your automation.

	\item As a library for fully-automated grading. Pretor provides
		powerful primitives that could be used as the basis for an
		unattended grading system. The interactive grading REPL also
		supports the execution of script files, allowing it to be run
		in a headless unattended mode.

\end{enumerate}

There are three major components of Pretor:

\paragraph{\texttt{pretor-psf}} is used by students to generate PSFs (Pretor
Submission Files), which they can submit through whatever mechanism you find
appropriate.

\paragraph{\texttt{pretor-grade}} implements an interactive REPL that enables a
grader to efficiently iterate through many PSFs in sequence.
\texttt{pretor-grade} ultimately produces more PSFs as output, which have the
grades and other feedback the grader assigns "burned in" to them.

\paragraph{\texttt{pretor-export}} is a tool which operates on the PSFs
produced by \texttt{pretor-grade} and generates output files that can be read
by humans, or imported by an LMS for bulk grading.

\section{Understanding the Pretor Data Model}

\begin{figure}[H]

	\centering

	\resizebox{0.8\textwidth}{!}{\digraph{abc}{
		rankdir=LR
		Grade -> Assignment;
		Assignment -> Course;
		Course -> Assignment;
		PSF->Revision;
		Revision->Grade;
		PSF -> "Parent Revision";
		"Parent Revision" -> Revision;
	}}

	\caption{High-level overview of the Pretor data model
	\label{fig:datamodel}}

\end{figure}

Understanding Pretor's data model is critical to make efficient use of its
features. Fortunately, Pretor has a relatively simple data model that aligns
closely with how courses, grades, and submissions are intuitively reasoned
about.

\begin{itemize}

	\item A PSF contains one or more revisions.

	\item A revision may contain zero or one grades.

	\item A grade is associated with an assignment.

	\item An assignment is associated with a course.

	\item A course is associated with one or more assignments.

\end{itemize}

\pretoremph{\textbf{Aside}: for technical reasons, all PSF files contain
serialized copies of the assignment and course information for each grade they
contain. This is because a grade is meaningless without a rubric (to weight
each category score), and a course (to determine the weight of the assignment
overall).}

You as the instructor interact with the data model in several ways. One
important way is by writing a \textbf{course definition file}, which defines
the set of assignments in your course and their relative weights, as well as
the rubric categories for each assignment. This is used by
\texttt{pretor-grade} to compute the scores for each assignment you grade, and
by \texttt{pretor-export} to generate appropriate score values.

You will also interact with the data model by grading assignments. Each time
you grade an assignment, you are creating a \textbf{revision} in the PSF the
student turned in, creating a \textbf{grade}, and attaching the grade to the
revision.

\pretoremph{\textbf{Key Concept}: Pretor has it's own internal revision control
system. Student-generated PSFs contain an initial "submission" revision. When
you grade a PSF, you create a "grade" revision, which can include both changes
to score or other metadata, as well as changes to the student's submitted code.
This is valuable because it makes it easy to track changes made to get student
code to compile, and allows you to make in-line comments within the student's
code. You can even revise an existing grade revision later if you realize you
made a mistake, which would create a third revision. Arbitrarily many grade
revisions may be made.}

\section{Pretor Workflow}

Using Pretor is straightforward, barring additions made by third-party plugins,
a typical Pretor grading workflow looks like this:

\begin{enumerate}

	\item Download PSFs for a specific assignment from your institution's
		LMS

	\item Run \texttt{pretor-grade} on the downloaded files, assigning a
		grade to each, this produces more PSFs which contain both the
		student's original submission and your modifications and
		feedback

	\item Run \texttt{pretor-export} on the PSFs generated in the previous
		steps to generate a CSV file appropriate for upload into your
		LMS

\end{enumerate}

\chapter{Grading With Pretor}

\section{Grading Basics}

You can begin an interactive grading session with the command
\texttt{pretor-grade}. \texttt{pretor-grade} has many useful parameters you
should explore\footnote{see \texttt{pretor-grade -{}-help}}, but the most
important are \texttt{-{}-ingest}, \texttt{-{}-outputdir}, and
\texttt{-{}-coursepath}.

\paragraph{\texttt{-{}-ingest}} is used to specify a directory where you have
downloaded your PSFs. This directory is searched recursively for
\texttt{*.psf}, all of which are loaded into your grading REPL before it
begins. You can also ingest PSFs after launching via the \texttt{ingest}
command.

\paragraph{\texttt{-{}-outputdir}} when you finish grading an assignment and mark
it as finalized, the resulting PSF will be stored in this directory. If
unspecified, they'll be placed in your working directory.

\paragraph{\texttt{-{}-coursedir}} specifies the directory where your course
definition file(s) are stored. When you begin grading a PSF, the course name
and assignment name specified in the submission's \texttt{pretor.toml} are
looked up by recursive search through every TOML in your configured coursedir.
When a matching file is found, it is loaded and used to pre-populate the
\texttt{grade.toml} that you will use to enter your scores. If you don't
specify this, your working directory will be used.

You should be greeted by a prompt that looks like this:

\begin{verbatim}
PRETOR version 0.0.1 interactive grading shell.
grader>
\end{verbatim}

You can enter the \texttt{help} command here to see a list of all commands
available in the REPL, and \texttt{help <command name>} to see documentation
for a specific command.

While there are many useful commands available, the most essential are:

\paragraph{\texttt{loaded}} displays a list of PSF files that have been loaded

\paragraph{\texttt{current}} show information about the PSF you are working on
right now

\paragraph{\texttt{next}} load the next un-graded PSF that is loaded

\paragraph{\texttt{interact}} drop to a Bash shell to grade the PSF

\paragraph{\texttt{showgrade}} show the score card for the current PSF

\paragraph{\texttt{finalize}} save your changes to the PSF and write it out
into the configured output directory

With only these commands, you can perform all grading tasks with Pretor.

Let's look at an example grading session with Pretor:

\begin{verbatim}
$ pretor-grade --ingest submissions
INFO: Loading PSF file 'submissions/Spring 1973-ABC123-2-jsmith-Assignment 1.psf'
INFO: Loading PSF file 'submissions/Spring 1973-ABC123-2-jdeer-Assignment 1.psf'
INFO: Loading PSF file 'submissions/Spring 1973-ABC123-2-jdoe-Assignment 1.psf'
PRETOR version 0.0.1 interactive grading shell.
grader> loaded
       0: Spring 1973-ABC123-2-jsmith-Assignment 1.psf
       1: Spring 1973-ABC123-2-jdeer-Assignment 1.psf
       2: Spring 1973-ABC123-2-jdoe-Assignment 1.psf
       grader> next
<PSF ID=UUID('7c6f7597-aba0-43bd-bee6-a829943bfcd7')>
semester      Spring 1973
section       2
assignment    Assignment 1
group         jsmith
course        ABC123
timestamp     2019-02-06 19:30:33.046311
archive_name  submissions/Spring 1973-ABC123-2-jsmith-Assignment 1.psf
PSF has NOT been graded
grader> interact
INFO: dropping you to a shell: bash --norc
grading Assignment 1 by jsmith $ tree
.
├── grade.toml
└── submission
    ├── doc
    │   └── HOWTO.txt
    ├── hello.c
    ├── Makefile
    ├── pretor.toml
    ├── util.c
    └── util.h

2 directories, 7 files
grading Assignment 1 by jsmith $ cat grade.toml
feedback = ""
bonus_multiplier = 0.0
bonus_marks = 0
bonus_score = 0.0
penalty_multiplier = 0.0
penalty_marks = 0
penalty_score = 0.0
assignment_name = "Assignment 1"

[categories]
correctness = 70
style = 30

\end{verbatim}

The \texttt{grade.toml} file is perhaps the most important thing to notice
here. This is how you input the grade you would like to assign. When you
interact with a PSF for the first time, this file is populated with the maximum
values for each category as determined by your course definition file. In other
words, every submission starts out with a 100\% score, and modifying the values
in the \texttt{[categories]} section allows you to change the submission's
score.

\begin{verbatim}
grading Assignment 1 by jsmith $ exit
exit
INFO: shell session terminated
grader> showgrade
SCORECARD FOR ABC123: Assignment 1

CATEGORY     MARKS  MAX MARKS  PERCENT SCORE
correctness  70     70         100.00%
style        30     30         100.00%

OVERALL MARKS: 100
MAXIMUM OVERALL MARKS: 100
RAW SCORE: 100.00%

OVERALL SCORE: 100.00%

grader> finalize
INFO: writing to 'Spring 1973-ABC123-2-jsmith-Assignment 1.psf'
grader> exit
$ pretor-psf --scorecard --input Spring\ 1973-ABC123-2-jsmith-Assignment\ 1.psf
SCORECARD FOR ABC123: Assignment 1

CATEGORY     MARKS  MAX MARKS  PERCENT SCORE
correctness  70     70         100.00%
style        30     30         100.00%

OVERALL MARKS: 100
MAXIMUM OVERALL MARKS: 100
RAW SCORE: 100.00%

OVERALL SCORE: 100.00%

\end{verbatim}

Notice that the assigned grade is saved out to disk as soon as
\texttt{finalize} is issued, and can be retrieved later using
\texttt{pretor-psf}.

Now consider an example where we have graded several PSFs already, and want to
see the overall score on each. For this, we can use the \texttt{pretor-export}
command:

\begin{verbatim}
$ ls
 coursedefs        'Spring 1973-ABC123-2-jdeer-Assignment 1.psf'    submissions
 README.md         'Spring 1973-ABC123-2-jdoe-Assignment 1.psf'
 sample_solutions  'Spring 1973-ABC123-2-jsmith-Assignment 1.psf'
 $ pretor-export --input '*.psf' --table
Spring 1973  ABC123  2  jsmith  90
Spring 1973  ABC123  2  jdeer   50
Spring 1973  ABC123  2  jdoe    90  submitted late, -10%
\end{verbatim}

\section{Bonuses, Penalties \& Grade Calculation}

Each grade contains a number of \textbf{categories}. A category has a maximum
number of marks\footnote{Maximum marks on a category is determined by the
associated course and assignment} and a number of assigned marks. A grade's raw
score is simply
$\frac{\sum\text{marks}_\text{max}}{\sum\text{marks}_\text{assigned}}$. The
function of \textbf{categories} is to allow the instructor to provide greater
granularity to assigned scores, and to accurate codify the categories of
an assignment's rubric.

\begin{wrapfigure}{l}{0.5\textwidth}

	\centering

	\begin{tabular}{c | c | c}

		category & $\text{marks}_\text{max}$ & $\text{marks}_\text{assigned}$ \\
		\hline\hline
		all test cases pass & 50 & 40 \\
		\hline
		correct style & 30 & 25 \\
		\hline
		code is documented & 30 & 30 \\

	\end{tabular}

	\caption{\label{fig:category_example} Example categories and assigned
	scores}

\end{wrapfigure}

Considering the example shown in figure \ref{fig:category_example}, the maximum
raw marks for this grade would be $50+30+30=110$, and the assigned marks would
be $40+25+30=95$, for a \textbf{raw score} of $\frac{95}{110} \approx 0.863$.
Note that the total number of maximum marks does not need to sum to 100 (this
example was deliberately constructed to demonstrate this).

While bonuses may be given by simply entering marks higher than the maximum for
a given category, this is not the suggested approach, as Pretor includes
dedicated facilities for specifying bonuses (and penalties). The final score of
a given grade is computed as: $g = \frac{m + b_m - p_m}{M} \cdot (1.0 + b - p)
+ B - P$ where:

\begin{itemize}

	\item $g$ is the final percent score in 0..1 (scores of higher than 1
		may be possible with bonus)

	\item $m$ is the number of earned marks on the assignment (sum of
		category scores)

	\item $b_m$ is the number of bonus marks on the assignment

	\item $p_m$ is the number of penalty marks on the assignment

	\item $M$ is the maximum number of marks on the assignment (sum of
		category maxes)

	\item $b$ is the bonus multiplier

	\item $p$ is the penalty multiplier

	\item $B$ is the score bonus

	\item $P$ is the score penalty

\end{itemize}

The \texttt{grade.toml} file generated while \texttt{interact}-ing with a PSF
with \texttt{pretor-grade} will automatically have appropriate fields for each
type of bonus and penalty initialized to values that will provide no bonus and
no penalty.

Finally, to handle cases where the provided facilities are insufficient for
assigning the desired grade, a \texttt{grade.toml} file may also specify a
\texttt{override} field, which if provided, is unconditionally used as the
final grade. Note that the \texttt{override} field is on a scale of $0..1$,
such that a value of $1$ would be a $100\%$ score, although \texttt{override}
is not bounded by $0..1$ (i.e. scores above $100\%$ or lower than $0\%$ may be
assigned. This convention is used throughout Pretor except where noted.

\pretoremph{\textbf{Hint:} Multiple penalties and bonuses can be
mix-and-matched. For the sake of clarity, the example shown here only shows one
penalty and one bonus applied at a time.}

An example grading session is shown below demonstrating assigning bonuses,
penalties, and overrides:

\subsection{Example Grading Session}

\subsubsection{Initial Grade}

\begin{verbatim}
$ pretor-grade --ingest submissions --coursepath coursedefs
./
INFO: Loading PSF file 'submissions/Spring 1973-ABC123-2-cad-Assignment 1.psf'
PRETOR version 0.0.3-dev interactive grading shell.
grader> next
<PSF ID=c1a0b0a3-1972-4974-a890-25f97c270b4c>
semester        Spring 1973
section         2
assignment      Assignment 1
group           cad
course          ABC123
timestamp       2019-02-10 14:02:18.709983
pretor_version  0.0.3-dev
archive_name    submissions/Spring 1973-ABC123-2-cad-Assignment 1.psf
PSF has NOT been graded
grader> interact
INFO: dropping you to a shell: bash --norc
grading Assignment 1 by cad $ vim grade.toml
grading Assignment 1 by cad $ cat grade.toml
feedback = ""
bonus_multiplier = 0.0
bonus_marks = 0
bonus_score = 0.0
penalty_multiplier = 0.0
penalty_marks = 0
penalty_score = 0.0
assignment_name = "Assignment 1"

[categories]
correctness = 50
style = 20
grading Assignment 1 by cad $ exit
exit
INFO: shell session terminated
grader> showgrade
SCORECARD FOR ABC123: Assignment 1

CATEGORY     MARKS  MAX MARKS  PERCENT SCORE
correctness  50     70         71.43%
style        20     30         66.67%

OVERALL MARKS: 70
MAXIMUM OVERALL MARKS: 100
RAW SCORE: 70.00%

OVERALL SCORE: 70.00%
\end{verbatim}

\subsubsection{Assigning a Bonus}

\begin{verbatim}
grader> interact
INFO: dropping you to a shell: bash --norc
grading Assignment 1 by cad $ vim grade.toml
grading Assignment 1 by cad $ cat grade.toml
feedback = ""
bonus_multiplier = 0.0
bonus_marks = 10
bonus_score = 0.0
penalty_multiplier = 0.0
penalty_marks = 0
penalty_score = 0.0
assignment_name = "Assignment 1"

[categories]
correctness = 50
style = 20
grading Assignment 1 by cad $ exit
exit
INFO: shell session terminated
grader> showgrade
SCORECARD FOR ABC123: Assignment 1

CATEGORY     MARKS  MAX MARKS  PERCENT SCORE
correctness  50     70         71.43%
style        20     30         66.67%
BONUS MARKS  10     --         --

OVERALL MARKS: 70
MAXIMUM OVERALL MARKS: 100
RAW SCORE: 70.00%
RAW SCORE NET OF BONUS/PENALTY MARKS: 80.00%

OVERALL SCORE: 80.00%
\end{verbatim}

\subsubsection{Assigning a Penalty}

\begin{verbatim}
grader> interact
INFO: dropping you to a shell: bash --norc
grading Assignment 1 by cad $ vim grade.toml
grading Assignment 1 by cad $ cat grade.toml
feedback = ""
bonus_multiplier = 0.0
bonus_marks = 10
bonus_score = 0.0
penalty_multiplier = 0.1
penalty_marks = 0
penalty_score = 0.0
assignment_name = "Assignment 1"

[categories]
correctness = 50
style = 20
grading Assignment 1 by cad $ exit
exit
INFO: shell session terminated
grader> showgrade
SCORECARD FOR ABC123: Assignment 1

CATEGORY     MARKS  MAX MARKS  PERCENT SCORE
correctness  50     70         71.43%
style        20     30         66.67%
BONUS MARKS  10     --         --

OVERALL MARKS: 70
MAXIMUM OVERALL MARKS: 100
RAW SCORE: 70.00%
RAW SCORE NET OF BONUS/PENALTY MARKS: 80.00%

PENALTY MULTIPLIER: 0.10
SCORE NET OF BONUS/PENALTY MULTIPLIER: 72.00%

OVERALL SCORE: 72.00%
\end{verbatim}

\subsubsection{Assigning an Override}

\begin{verbatim}
grader> interact
INFO: dropping you to a shell: bash --norc
grading Assignment 1 by cad $ vim grade.toml
grading Assignment 1 by cad $ cat grade.toml
feedback = ""
bonus_multiplier = 0.0
bonus_marks = 10
bonus_score = 0.0
penalty_multiplier = 0.10
penalty_marks = 0
penalty_score = 0.0
assignment_name = "Assignment 1"
override = 0.9

[categories]
correctness = 50
style = 20
grading Assignment 1 by cad $ exit
exit
INFO: shell session terminated
grader> showgrade
SCORECARD FOR ABC123: Assignment 1

CATEGORY     MARKS  MAX MARKS  PERCENT SCORE
correctness  50     70         71.43%
style        20     30         66.67%
BONUS MARKS  10     --         --

OVERALL MARKS: 70
MAXIMUM OVERALL MARKS: 100
RAW SCORE: 70.00%
RAW SCORE NET OF BONUS/PENALTY MARKS: 80.00%

PENALTY MULTIPLIER: 0.10
SCORE NET OF BONUS/PENALTY MULTIPLIER: 72.00%

SCORE HAS BEEN OVERRIDDEN BY GRADER

OVERALL SCORE: 90.00%
\end{verbatim}

\section{Constructing \texttt{pretor.toml} Files}

When a student generates a PSF from a project directory, \texttt{pretor-psf}
uses the \texttt{pretor.toml} file in the top-level directory of the project to
"burn in" metadata such as the assignment, course, semester, and section
number. While this information can also be supplied using command-line
arguments\footnote{see \texttt{pretor-psf -{}-help}}, the use of
\texttt{pretor.toml} reduces the potential for human error which is important
given that metadata fields are often matched using exact string comparison.

Typically, the instructor for a course will write a \texttt{pretor.toml} file
for each assignment (and possibly for each section\footnote{According to the
needs of the specific course, the section number/identifier may be provided via
\texttt{pretor.toml} or via \texttt{pretor-psf -{}-section}}. The
\texttt{pretor.toml} file may either be provided to students with an assignment
template, or by asking students to download and install it into their projects.

A \texttt{pretor.toml} file may define any of the following fields:

\begin{itemize}

	\item \texttt{exclude} -- a list of glob patterns to exclude from being
		included in the generated PSF

	\item \texttt{course} -- the string name of the course, i.e. "CS101"

	\item \texttt{section} -- the section identifier\footnote{Note that the
		section identifier does not have to be numeric, although
		it is occasionally referred to as such. The section identifier
		is stored as a string to avoid constraining University section
		numbering conventions}

	\item \texttt{semester} -- the string name of the semester, i.e.
		"Spring 1994"

	\item \texttt{assignment} -- the string name of the assignment, i.e.
		"Assignment 1"

	\item \texttt{minimum\_version} -- the minimum Pretor version which
		can be used to pack this assignment as a string, i.e. "0.0.2"

\end{itemize}

\subsection{A Sample \texttt{pretor.toml}}

\begin{verbatim}
assignment = "Assignment 1"
section = 2
semester = "Spring 1973"
course = "ABC123"
exclude = ["*.o"]
\end{verbatim}

\subsection{Bypassing Metadata Checks}

\pretoremph{ \textbf{Danger Zone}: The information in this section will allow
you to generate PSFs with missing or incorrect metadata which may impede
grading. }

\pretoremph{ \textbf{Students Beware}: As this documentation is distributed
publicly, it is entirely possible that student users of Pretor may stumble upon
this information. The procedures described here are deliberately hidden from
the "help" information of \texttt{pretor-psf} to protect you from "shooting
yourself in the foot". Using these procedures has the potential to make your
submissions significantly more difficult to grade for your instructor, which is
unlikely to endear you to them. Consider yourself warned.}

Pretor features three checks to help prevent user error on the part of students
while packing assignments to PSF for submission. These may be disabled via
(intentionally) undocumented options to \texttt{pretor-psf}. In some cases, it
may be necessary for an instructor, developer, or administrator to bypass one
or more of these checks. The correct procedures to do so are documented below.

\paragraph{Metadata Check} This check requires that all of the metadata fields
"semester", "section", "assignment", "group", and "course" are specified either
via command-line arguments, or via \texttt{pretor.toml}. It may be disabled via
the flag \texttt{-{}-no\_meta\_check}. Specifying this flag will assert the key
\texttt{no\_meta\_check} in both the forensic information and metadata of the
generated PSF.

\paragraph{\texttt{pretor.toml} Check} This check requires that the top level
directory of the submission directory contains a \texttt{pretor.toml} file.
This is intended to catch cases of users accidentally specifying the wrong
directory to generate a PSF from. This check may be disabled using
\texttt{-{}-allow\_no\_toml}, doing so will assert the key
\texttt{allow\_no\_toml} in both the forensic information and metadata of the
generated PSF.

\paragraph{Version Check} This check allows an instructor to specify a minimum
Pretor version to be used for packing the assignment via the
\texttt{minimum\_version} key of \texttt{pretor.toml}. This is so that if a
future Pretor version adds a new feature that is needed to correctly pack the
submission, student users will not accidentally use an outdated version.  This
check may be bypassed using the flag \texttt{-{}-disable\_version\_check},
which will assert the field \texttt{disable\_version\_check} in both the
forensic information and metadata of the generated PSF.

\section{Constructing Course Definitions}

As the instructor for a course, you will need to write a course definition file
for your course. This codifies the set of assignments (or other graded
materials), their relative weights within the course, and their rubrics.

For a Pretor grade to be meaningful, it must be associated with a
\textbf{course definition}. When you \texttt{interact} with a PSF in
\texttt{pretor-grade}, the course name field (\texttt{course}) of it's
\texttt{pretor.toml} file is searched for among all course definition files in
the specified course directory for one with a matching \texttt{name} field.

\pretoremph{\textbf{Note:} When you grade a PSF with \texttt{pretor-grade}, the
course definition used at the time is "burned in" to the output PSF. This is to
ensure that future modifications to the course definition will not
retro-actively change existing grades.  }

A course definition file may have any name the author finds descriptive, but
must have the extension \texttt{.toml}. At a minimum a course definition must
define a section named \texttt{course} containing a key named \texttt{name},
which must specify the course name (which is matched against the
\texttt{course} in \texttt{pretor.toml} files). The \texttt{course} section may
also optionally contain a \texttt{description} field, which is an arbitrary
human-readable string for reference purposes.

All other sections in the course definition file may have arbitrary names, and
correspond to assignment rubrics. Each such section must define a \texttt{name}
field which is matched against the \texttt{assignment} field in
\texttt{pretor.toml} files, as well as a floating point \texttt{weight} field
defining the assignment's weight\footnote{Course definition \texttt{weight}
fields assume that a perfect score in the course is $1.0$.}. An optional
\texttt{description} field may be defined as well. Finally, each additional
field specified is assumed to define the maximum number of marks on a rubric
category.

\pretoremph{\textbf{Best Practice}: is to store all of your course definition
files in a single directory, so you can easily reference them via
\texttt{pretor-grade}.}

\subsection{A Sample Course Definition File}

\begin{verbatim}
[course]
name = "ABC123"
description = "Imaginary course for testing Pretor."

[assignment_1]
name = "Assignment 1"
description = "Description for the first assignment"
weight = 0.05
correctness = 70
style = 30

[assignment_2]
name = "Assignment 2"
description = "Description for the second assignment"
weight = 0.1
correctness = 70
style = 30

[assignment_3]
name = "Assignment 3"
description = "Description for the third assignment"
weight = 0.1
correctness = 70
style = 30

[midterm]
name = "Midterm"
weight = 0.2
problem_1 = 10
problem_2 = 10
problem_3 = 10
problem_4 = 70

[quiz1]
name = "Quiz 1"
weight = 0.1
problem_1 = 10
problem_2 = 10
problem_3 = 10

[quiz2]
name = "Quiz 2"
weight = 0.1
problem_1 = 10
problem_2 = 10
problem_3 = 10
problem_4 = 10
problem_5 = 10

[final]
name = "Final Exam"
weight = 0.35
problem_1 = 10
problem_2 = 10
problem_3 = 10
problem_4 = 30
problem_5 = 20
problem_6 = 20
problem_7 = 10
\end{verbatim}

\section{Advanced Grading REPL Topics}

\pretoremph{\textbf{Note}: The full array of commands available in the grading
REPL is not documented here, you can use \texttt{help} to view a list of
commands or \texttt{help <command>} to view documentation for a specific
command.}

The Pretor grading REPL (accessed via \texttt{pretor-grade}) is in fact a
robust domain-specific language for interacting with Pretor's internal data
structures. Although it is intentionally not Turing-complete\footnote{It is in
fact possible that the grading REPL is Turing-complete, but this has been
deliberately not investigated, as it is not intended to be. Don't turn your
grading workflow into a Turing tarpit. You have been warned}, the grading
REPL contains many useful features including:

\begin{itemize}

	\item A user readable and writable symbol table for handling
		runtime configuration (an entry in the symbol table is
		conceptually equivalent to a variable).

	\item A convenient method to view the REPL's internal state -- via
		rad-only entries in the symbol table.

	\item History (via the \texttt{history} command) and tab completion.

	\item Capability to execute shell commands by prefixing them with
		\texttt{!}, i.e. \texttt{!echo hello}.

	\item Basic error handling and recovery facilities.

\end{itemize}

\subsection{The Symbol Table}

The symbol table can be directly interacted with via three main commands:

\begin{itemize}

	\item \texttt{symtab} -- display all symbols in the symbol table and
		their current values. Note that symbols prefixed with
		\texttt{\#} are read-only, and are generally used internally for
		book-keeping by the REPL.

	\item \texttt{get} -- get the value of a single symbol (i.e.
		\texttt{get symname}).

	\item \texttt{set} -- set the value of a single symbol (i.e.
		\texttt{set symname newval}). The old value is returned as the
		result of the command.

\end{itemize}

The symbols in the symbol table are documented in figure \ref{fig:symtab}. Note
that additional symbols may be added by custom RC scripts or via plugins.

\begin{figure}[H]

	\centering

	\begin{tabular}{p{0.2\textwidth} | p{0.7\textwidth}}

		Symbol & Purpose \\ \hline\hline

		\texttt{\#result} & The result of the most recently
		executed command, this is displayed to the console after the
		command finishes running. \\ \hline

		\texttt{\#lastresult} & The result of the second most recently
		executed command. \\ \hline

		\texttt{\#status} & True if the command finished successfully,
		otherwise False. \\ \hline

		\texttt{\#lastatatus} & The status for the previous command. \\
		\hline

		\texttt{\#error} & The error text if an error occurred by the
		command.  \\ \hline

		\texttt{\#finalized} & A list of indices of \#psf which have
		been finalized already \\ \hline

		\texttt{\#argv} & Split list of arguments to the current
		command. \\ \hline

		\texttt{\#psf} & A list of currently loaded PSF objects. \\
		\hline

		\texttt{coursepath} & \texttt{:}-delimited list of directories
		to search for course definition files in. \\ \hline

		\texttt{outputdir} & The directory where finalized PSFs are
		written to. \\ \hline

		\texttt{revision} & The revision to interact with when using
		the \texttt{interact} command. If empty, an appropriate value
		is generated automatically. This may be overridden to inspect
		a specific revision in a PSF. \\ \hline

		\texttt{base\_revision} & The revision to use as the base
		ungraded revision submitted by the student. This usually does
		not need to be modified \\

	\end{tabular}

	\caption{\label{fig:symtab} Table of symbols and their purposes.}

\end{figure}

\subsection{Writing a RC File}

By default, all commands stored in \texttt{~/.config/pretor/rc} are executed in
order when the REPL starts up. The file used for this purpose may be overridden
via \texttt{pretor-grade -{}-rc}. Lines beginning with the \texttt{\#}
character are ignored to facilitate leaving comments.

This file is useful for setting up configuration options that will be used on a
regular basis. For example, the course definition search path, or finalized PSF
output directory might be defined here to avoid needing to specify these for
each grading session.

\section{Exporting Grades}

Grades may be exported for upload to a LMS\footnote{Learning Management System,
such as Moodle, Blackboard, et. al.}. This is accomplished via the
\texttt{pretor-export} command. This command iterates over a collection of PSF
files and generates output in one of several formats. For full documentation,
see \texttt{pretor-export -{}-help}.

The Moodle formatted output\footnote{Obtained via the \texttt{-{}-moodle}
flags}, which is simply a text CSV format, is likely the most amenable to the
application of your own automation, and should be well suited for parsing with
tools such as Excel, LibreOffice, or
xsv\footnote{\url{https://github.com/BurntSushi/xsv}}.

\subsection{Sample Usage of \texttt{pretor-export}}

\begin{verbatim}
$ pretor-export --input *.psf --moodle
"Spring 1973","ABC123",2,"cad",100.0,""
$ pretor-export --input *.psf --table
Spring 1973  ABC123  2  cad  100
$ pretor-psf --scorecard --input Spring\ 1973-ABC123-2-cad-Assignment\ 1.psf
SCORECARD FOR ABC123: Assignment 1

CATEGORY     MARKS  MAX MARKS  PERCENT SCORE
correctness  70     70         100.00%
style        30     30         100.00%

OVERALL MARKS: 100
MAXIMUM OVERALL MARKS: 100
RAW SCORE: 100.00%

OVERALL SCORE: 100.00%
\end{verbatim}

\chapter{The Pretor Submission File (PSF) Format}

\section{File Structure}

\section{Forensic Information}

\section{Hand-Editing Pretor Submission Files}

\chapter{Pretor for Systems Administrators}

\section{Deploying Pretor with \texttt{pip}}

\section{Deploying Pretor with \texttt{pyinstaller}}



\end{document}
